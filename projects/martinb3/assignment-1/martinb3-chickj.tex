\documentclass[12pt]{article}
\usepackage{times}
\usepackage{cite}
%this is a comment
\title{Vision Statement}
\author{Jesse Chick & Benjamin Martin}




\begin{document}
\maketitle
\tableofcontents



\section{Problem}
Every term, thousands of OSU students are dismayed by the often high prices they pay at the Beaver Store for their prescribed textbooks. In addition to an unappealing sticker price, students experience the latent annoyance at the conclusion of each ten week term when they are faced with a stack of expensive, glossy, barely touched books for which they have no concrete use. 
Regardless of the monetary and material waste, blasphemous to anyone concerned with preserving something resembling frugality in a hopelessly consumerist culture, the facility for the undesirable situation faced by current students is not difficult to pin down. In acquiring textbooks, students are limited to a few main sources: online retailers, the publisher itself, and the good ol’ Beaver Store. The latter has the advantage of providing buyers with a physical specimen which is guaranteed to be the correct book for the class and of the correct edition. The buyer experiences the immediacy of walking out of the store with their merchandise in hand, as opposed to an indefinite shipping period. In lieu of the aggressive savvy to get my textbooks through a less expensive (and thus less accessible) mean, I myself succumb to the convenience of the Beaver Store’s offerings, with predictable frustrations. And I am by no means the only one in this boat.



\section{Proposed Solution}


We propose a solution which begins to solve the issue of high cost and the waste of a perfectly good textbook: a mobile application which allows users (OSU students) to buy and sell used textbooks. Our users will be able to enter essential information regarding their used textbooks (title, ISBN, a price of their choosing) with additional optional fields (edition, publisher, associated course(s)), which will be viewable to all other users. Users can search for textbooks by ISBN, compare prices, and connect with the owners of the book they wish to purchase.
We want to create an ios application that can help college students exchange and acquire their textbooks more easily for classes.
This application will provide the user with ways to link their bank account to their phone and then be able to use their money to trade for books or sell their old books.
The app would detect other students that have the book that the student would need for their courses for next term and check for pricing ranges and the best deals.
Our application will inherently promote more affordable prices for textbooks because in order to be able to sell a textbook through the app, a textbook owner will have to name a price substantially below the Beaver Store asking price. Likewise, this asking price must be higher than what they would get by selling it back to the Beaver Store for it to be worth the seller’s time. This means that the prices of textbooks on our app will be somewhere in between the Beaver Store price of the textbook and the amount the Beaver Store is willing to buy the book back for. Everyone wins: the seller gets more money for their textbook than they would have otherwise, and the buyer pays a lower price than they would buying through usual means. These benefits will encourage students to pass on their old textbooks rather than allowing them to accumulate dust on a bookshelf, meaning that, collectively, OSU students will be able to get more more learning out of less printed paper. 
What separates this from Craigslist or similar resources is its exclusivity. This application, at least at its conception, will be for OSU students only (they will be required login with a valid OSU email) which we hope will foster a welcoming, safe user experience. Similarly, our application has the benefit of exclusively marketing textbooks and academic media, so it is more likely that customers will be able to find what they are looking for if it is listed. 
This project will take the form of a mobile application. In order to do everything outlined above, we will need several cellular devices for testing, as well a database in which to store textbook information and a server of some description to serve such information. The markup language- and scripting skills required to build an attractive, functional application are taught in CS290. 

\section{Limitations}

Linking bank accounts would have to be very secure since this is very important information and would undoubtedly be the hardest thing to implement in this particular project. A key limitation in our proposal is that it will be difficult (if possible, given time constraints) to achieve a strong standards when it comes to verifying valid OSU emails. Checking for the presence of “@oregonstate.edu” in the input is a good start, but we will have to attempt to connect with that email in some way to ensure that it exists, a process which we have no concrete knowledge of at the present time, though it does not seem the least bit infeasible. 


Textbook price increases~\cite{bookhuff}
Even more textbook increase~\cite{booknbc}


\bibliography{myref}
\bibliographystyle{plain}

\end{document}
